\problemname{Tree Hugging}
% (also known as inosculation, apparently)

\noindent
Once, two trees forgot their place and started to grow into each other.
One of the trees grew from the left, and the other from the right.
On $n$ points, they collided.

Numbering the points $1, 2, \dots, n$ from left to right,
the left tree ended up connecting all of them in a single subtree
rooted in node $1$, such that
every node's children had larger numbers than the node itself.
We can describe this subtree with a list of $n-1$ edges.

Similarly, the right tree also connected all nodes in a single subtree
rooted in node $n$, with
every node's children having smaller numbers than the node itself.
This yields an additional $n-1$ edges.

Now, given the full list of $2(n-1)$ edges, it is not necessarily easy to
tell which edge belongs to which tree.
Can you figure out a possible assignment, or determine that it is impossible
for this collection to have been the union of two trees?

\section*{Input}
The first line of input contains the integer $n$ ($2 \le n \le 10^5$).
The next $2(n-1)$ lines each contain two integers $u, v$ ($1 \le u < v \le n$) indicating an edge joining the two nodes $u$ and $v$.
A pair $(u, v)$ may be connected by more than one edge.

\section*{Output}
If it is possible for the edges to be the union of two trees that
grow left-to-right and right-to-left, output a string of length $2(n-1)$,
where the $i$'s character
is \texttt{L} if the $i$'th edge should come from the left tree,
or \texttt{R} if it should come from the right tree.
Otherwise, output the word ``\texttt{impossible}'' on a single line.
If there are multiple solutions, you may output any one of them.

\section*{Explanation of Sample Inputs}
In the first example, there are two solutions: \texttt{LLRRRRLL} and \texttt{LLRLRRLR}.

In the second example, there are no solutions.
Note that \texttt{LRLR} is not valid, because it would involve the right tree
growing backward, from left to right.

\problemname{Color Codes}

\illustration{0.4 }{wall}{\href{https://www.flickr.com/photos/designmilk/23965668670}{Picture} by designmilk on Flickr, cc by-sa}%
\noindent
\emph{Gray codes} are a classic topic in information theory with a
number of practical applications, none of which we are concerned with
in this problem.  An $n$-bit Gray code is an ordering $(x_1,
x_2, \ldots, x_{2^n})$ of all $n$-bit binary strings, with the
property that any consecutive pair of strings differ in exactly $1$
bit.  More formally, for every $1 \le i < 2^n$, it holds that
$d(x_{i}, x_{i+1}) = 1$, where $d(\cdot, \cdot)$ denotes the Hamming
distance between two binary strings.  For instance, for $n=3$, the
sequence $(000, 001, 011, 010, 110, 111, 101, 100)$ is a Gray code.

While Gray codes are great, they are also a bit, well... gray\footnote{With
apologies to Frank Gray.}.  In this problem, we look at a much more
colorful variant.

For an integer $n \ge 1$ and set of integers $P \subseteq \{1,
\ldots, n\}$, we say that an ordering $(x_1, \ldots, x_{2^n})$ of all
$n$-bit binary strings is an \emph{$n$-bit color code with palette
  $P$}, if for all $1 \le i < 2^n$, it holds that $d(x_i, x_{i+1}) \in
P$, i.e., the number of bits by which any consecutive pair of strings
differ is in $P$.

Note that for some palettes, color codes do not exist.  For instance,
if $n = 6$ and $P = \{6\}$, the second string must be the binary
negation of the first one, but then the third string must be the
negation of the second one, i.e., equal to the first string.

Given $n$ and $P$, can you construct an $n$-bit color code with
palette $P$?


\section*{Input}
The first line of input consists of two integers $n$ ($1 \le n \le 16$)
and $p$ ($1 \le p \le n$).   Then follow a line with $p$ distinct integers
$s_1, \ldots, s_p$ ($1 \leq s_i \leq n$ for each $i$) -- the elements of $P$.

\section*{Output}

If there is an $n$-bit color code with palette $P$, output $2^n$
lines, containing the elements of such a code, in order.  If there are
many different codes, any one will be accepted.  If no such code
exists, output ``\texttt{impossible}''.

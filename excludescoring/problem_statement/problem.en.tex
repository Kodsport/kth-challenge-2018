\problemname{\#exclude<scoring>}
%% plainproblemname: #exclude<scoring>

\noindent
You are participating in a programming contest cup.  The cup consists
of a series of programming contests, followed by a final at the end of
the season for the $15$ top ranked contestants in the cup.  With only
one contest left to go before the final, you are starting to wonder if
your performance in the earlier contests has been good enough to
already secure you a spot in the finals.  If so, you could succumb to
your laziness and skip the last contest.

The ranking of the cup works as follows.  In each contest, a
contestant earns some number of points between $0$ and $101$ (the
details of this are described below).  Their \emph{aggregate score} is
then defined to be the \emph{sum of the four highest scores} achieved.
For instance if a contestant got $45$, $15$, $32$, $0$, $30$, and $20$
points over $6$ contests, their aggregate score is $45+32+30+20=127$.
The \emph{rank} of a contestant X \emph{in the cup} is defined to be
$1$ plus the number of contestants that have a strictly larger
aggregate score than X.

The score a contestant earns from a contest is based on the rank they
achieve \emph{in that contest}, according to the following table.

\begin{center}
\begin{tabular}{|c|c|||c|c|||c|c|}
    \hline
    \textbf{Rank} & \textbf{Points} &  \textbf{Rank} & \textbf{Points} & \textbf{Rank} & \textbf{Points} \\ \hline
    $1$ & $100$ & $11$ & $24$ & $21$ & $10$\\ \hline
    $2$ & $75$ & $12$ & $22$ & $22$ & $9$\\ \hline
    $3$ & $60$ & $13$ & $20$ & $23$ & $8$\\ \hline
    $4$ & $50$ & $14$ & $18$ & $24$ & $7$\\ \hline
    $5$ & $45$ & $15$ & $16$ & $25$ & $6$\\ \hline
    $6$ & $40$ & $16$ & $15$ & $26$ & $5$\\ \hline
    $7$ & $36$ & $17$ & $14$ & $27$ & $4$\\ \hline
    $8$ & $32$ & $18$ & $13$ & $28$ & $3$\\ \hline
    $9$ & $29$ & $19$ & $12$ & $29$ & $2$\\ \hline
    $10$ & $26$ & $20$ & $11$ & $30$ & $1$\\ \hline
    %% $1$ & $100$ & $16$ & $15$\\ \hline
    %% $2$ & $75$ & $17$ & $14$\\ \hline
    %% $3$ & $60$ & $18$ & $13$\\ \hline
    %% $4$ & $50$ & $19$ & $12$\\ \hline
    %% $5$ & $45$ & $20$ & $11$\\ \hline
    %% $6$ & $40$ & $21$ & $10$\\ \hline
    %% $7$ & $36$ & $22$ & $9$\\ \hline
    %% $8$ & $32$ & $23$ & $8$\\ \hline
    %% $9$ & $29$ & $24$ & $7$\\ \hline
    %% $10$ & $26$ & $25$ & $6$\\ \hline
    %% $11$ & $24$ & $26$ & $5$\\ \hline
    %% $12$ & $22$ & $27$ & $4$\\ \hline
    %% $13$ & $20$ & $28$ & $3$\\ \hline
    %% $14$ & $18$ & $29$ & $2$\\ \hline
    %% $15$ & $16$ & $30$ & $1$\\ \hline
\end{tabular}
\end{center}
\noindent
If a contestant gets a worse rank than $30$, they get $0$ points.  If
two or more contestants get the same rank in the contest, they are
instead assigned the average points of all the corresponding ranks.
This average is always rounded up to the closest integer.  For
example, if three contestants are tied for second place they all
receive $\lceil \frac{75 + 60 + 50}{3} \rceil = 62$ points, and the
next contestant will have rank $5$ and receives $45$ points (or less,
if there is a tie also for $5$'th place).  This applies also at rank
$30$, e.g., if $4\,711$ contestants are tied for $30$'th place, they
all receive $1$ point.

Contestants may participate in every contest either on-site or online.
If they compete on-site, they get $1$ extra point, no matter their
original number of points.  If a contestant does not participate in a
contest, they get $0$ points.


\section*{Input}

The first line of input contains two integers $n$ and $m$ ($2 \le n \le 10$, $1
\le m \le 10^5$), where $n$ is the number of contests in the cup
(excluding the final), and $m$ is the number of people who
participated in any of the first $n-1$ contests.

Then follow $m$ lines, each describing a contestant.  Each such line
consists of $n-1$ integers $0 \le s_1, \ldots, s_{n-1} \le 101$, where
$s_i$ is the score that this contestant received in the $i$th contest.

The first contestant listed is you.  The point values in the input
might not correspond to actual points from a contest.

\section*{Output}

Output a single integer $r$, the worst possible rank you might end up
in after the last contest, assuming you do not participate in it.
